%
% 6.006 problem set 1 solutions template
%
\documentclass[12pt,twoside]{article}

\input{macros-fa18}
\newcommand{\theproblemsetnum}{1}
\newcommand{\releasedate}{Tuesday, September 6}
\newcommand{\partaduedate}{Tuesday, September 13}

\title{6.006 Problem Set 1}

\begin{document}

\handout{Problem Set \theproblemsetnum}{\releasedate}
\textbf{All parts are due {\bf \partaduedate} at {\bf 11PM}}.

\setlength{\parindent}{0pt}
\medskip\hrulefill\medskip

{\bf Name:} Kevin Jiang

\medskip

{\bf Collaborators:} 

\medskip\hrulefill

%%%%%%%%%%%%%%%%%%%%%%%%%%%%%%%%%%%%%%%%%%%%%%%%%%%%%
% See below for common and useful latex constructs. %
%%%%%%%%%%%%%%%%%%%%%%%%%%%%%%%%%%%%%%%%%%%%%%%%%%%%%

% Some useful commands:
%$f(x) = \Theta(x)$
%$T(x, y) \leq \log(x) + 2^y + \binom{2n}{n}$
% {\tt code\_function}


% You can create unnumbered lists as follows:
%\begin{itemize}
%    \item First item in a list 
%        \begin{itemize}
%            \item First item in a list 
%                \begin{itemize}
%                    \item First item in a list 
%                    \item Second item in a list 
%                \end{itemize}
%            \item Second item in a list 
%        \end{itemize}
%    \item Second item in a list 
%\end{itemize}

% You can create numbered lists as follows:
%\begin{enumerate}
%    \item First item in a list 
%    \item Second item in a list 
%    \item Third item in a list
%\end{enumerate}

% You can write aligned equations as follows:
%\begin{align} 
%    \begin{split}
%        (x+y)^3 &= (x+y)^2(x+y) \\
%                &= (x^2+2xy+y^2)(x+y) \\
%                &= (x^3+2x^2y+xy^2) + (x^2y+2xy^2+y^3) \\
%                &= x^3+3x^2y+3xy^2+y^3
%    \end{split}                                 
%\end{align}

% You can create grids/matrices as follows:
%\begin{align}
%    A = 
%    \begin{bmatrix}
%        A_{11} & A_{21} \\
%        A_{21} & A_{22}
%    \end{bmatrix}
%\end{align}

% You can include images and PDFs as follows:
% \includegraphics[width=0.5\textwidth]{img.jpg}

\begin{problems}

\problem  % Problem 1

\begin{problemparts}
\problempart % Problem 1a

\begin{align*}
        f_1 &= 20n+18 = \Theta(n) \\
        f_2 &= 20n\cdot18 = \Theta(n) \\
   		f_3 &= 20n^{18} = \Theta(n^{18}) \\
        f_4 &= \log_{20}(n^{18}) = 18\log_{20}(n) = \Theta(\log(n)) \\
        f_5 &= (\log_{18}(n))^{20} = \Theta(\log(n)^{20}) \\                           
\end{align*}
$$ \Rightarrow (f_4, f_5, \{f_1, f_2\}, f_3)$$

\problempart % Problem 1b

\begin{align*}
	f_1 &= n^{2\log n} = \Theta(n^{\log n^2})\\
	f_2 &= 2^{2^{\log n}} = 2^n = \Theta(2^n) \\
	f_3 &= 2^{(\log n)^2} = (2^{\log n})^{\log n}= n^{\log n} =  \Theta(n^{\log n}) \\
	f_4 &= \Theta(n^{\log n}) \\
	f_5 &= \Theta(2^{\log (\log n)})
\end{align*}
$$ \Rightarrow (f_5, \{ f_3, f_4\}, f_1, f_2)$$

\problempart % Problem 1c

\begin{align*}
f_1 &= \Theta(2^{n^3}) \\
f_2 &= \Theta(2^{(n+1)^3}) \\
f_3 &= \Theta(n^{n^2}) \\
f_4 &= \Theta(4^{2^n}) \\
f_5 &= \Theta(3^{2^n}) \\
\end{align*}
$$ \Rightarrow (f_3,f_1,f_2, f_5, f_4)$$

\problempart % Problem 1d
\begin{align*}
f_1 &= (2n)! \approx \sqrt{4\pi n} \left(\dfrac{2n}{e}\right)^{2n} = \Theta\left(\left(\dfrac{2}{e}\right)^{2n}\cdot \sqrt{n}  \cdot n^{2n}\right)\\
f_2 &= \dfrac{(2n)!}{n!\cdot n!} \approx \dfrac{\sqrt{4\pi n} \left(\dfrac{2n}{e}\right)^{2n}}{2\pi n\cdot\left(\dfrac{n}{e}\right)^{2n}}  =  \Theta\left(\dfrac{1}{\sqrt{n}}\cdot 4^{n}\right)  \\
f_3 &= \Theta(4^n) \\
f_4 &= \Theta(2^n \cdot n^n) \\
f_5 &= \Theta((n^n)^2) \\
\end{align*}
$$ \Rightarrow (f_2, f_3, f_4, f_1, f_5)$$


\end{problemparts}

\newpage
\problem  % Problem 2

\begin{problemparts}
\problempart % Problem 2a
\begin{enumerate}[i.]
  \item % 1
  \item % 2
\end{enumerate}

\problempart % Problem 2b
\begin{enumerate}[i.]
  \item % 1 
  \item % 2
  \item % 3
  \item % 4
\end{enumerate}
\begin{center}
  % Here is an example of embedding images!
  \includegraphics[width=0.5\textwidth]{img.jpg}
\end{center}
\end{problemparts}

\newpage
\problem  % Problem 3

\begin{problemparts}
\problempart % Problem 3a
$\text{For each of the } n \text{ rows, we apply binary search to see if } v \text{ is present in the row. Since there are } m \text{ elements in each row, the running time for each of the } n \text{ rows will be }\log{m} \text{. Hence, in the worst case scenario where } v \text{ is not present in } A \text{, the algorithm is } O(n\log{m})$
\problempart % Problem 3b
\problempart % Problem 3c
\problempart Submit your implementation to {\small\url{alg.mit.edu}}.
\end{problemparts}

\end{problems}

\end{document}

